\documentclass[brazil]{article}

\usepackage[latin1]{inputenc}
\usepackage[portuguese]{babel}

\title{Prova de Programa\c{c}\~ao de Computadores}

\author{Prof.: Carlos Camar\~ao}

\date{6 de Julho de 2009}

\addtolength\textwidth{15mm}
\setlength{\parindent}{0pt}

\pagestyle{empty}

\begin{document}

\maketitle

\begin{enumerate}

\item Escreva um programa para determinar o vencedor de uma elei��o,
dados o n�mero de candidatos e os votos.

O primeiro valor da entrada � um inteiro $n$ representando o n�mero de
candidatos. Cada valor seguinte � um inteiro positivo entre 1 e $n$
que representa um voto (os candidatos s�o identificados por inteiros
positivos entre 1 e $n$).

O final dos dados � indicado por um voto igual a zero. Voc� pode supor
que os dados est�o corretos (o n�mero de candidatos e cada voto s�o
inteiros positivos). Cada inteiro da entrada � separado do seguinte
por um ou mais espa�os ou linhas.

A sa�da deve conter o n�mero do candidato que venceu (aquele que
obteve mais votos). Voc� pode supor que existe apenas um vencedor.

Exemplo: para a entrada:

{\tt 
\begin{tabular}{l}
4 1 2 1 4 3 0
\end{tabular} }

a sa�da deve ser:

{\tt 1}

\item A paridade de um inteiro $n$ � 0 se o n�mero de bits contidos na
sua representa��o bin�ria � par, caso contr�rio 0.

Por exemplo, 3 na base 2 � igual a 11, portanto tem paridade 0 (h� um
n�mero par de bits na sua representa��o bin�ria); 4 na base 2 � igual
a 100, portanto tem paridade 1 (h� um n�mero �mpar de bits na sua
representa��o bin�ria).

Escreva um programa que leia diversos n�meros positivos, separados por
espa�os ou linhas, e imprima, para cada valor, uma linha que cont�m a
sua representa��o bin�ria e a paridade desse valor.

O final da entrada � indicada por um valor 0 (zero).

Por exemplo: para a entrada:

{\tt \begin{tabular}{l}
1 2 10 21 0
\end{tabular}}

a sa�da deve ser:

{\tt \begin{tabular}{l}
A paridade de 1 � 1\\
A paridade de 10 � 1\\
A paridade de 1010 � 0\\
A paridade de 10101 � 1
\end{tabular}}

\end{enumerate}

\end{document}

