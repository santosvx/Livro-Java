\documentclass[brazil]{article}

\usepackage[latin1]{inputenc}
\usepackage[portuguese]{babel}

\title{Prova de Programa\c{c}\~ao de Computadores}

\author{Prof.: Carlos Camar\~ao}

\date{20 de Maio de 2009}

\addtolength\textwidth{15mm}
\setlength{\parindent}{0pt}

\pagestyle{empty}

\begin{document}

\maketitle

\begin{enumerate}

\item Escreva um programa que leia um inteiro positivo $n$ e imprima o
valor do somat�rio seguinte com $n$ parcelas:

    \[ \frac{1}{2} - \frac{2}{5} + \frac{3}{10} - \frac{4}{17} +
\frac{5}{26} - \ldots \]

Obs. 1: A primeira parcela � $\frac{1}{2}$ e cada parcela incrementa o
numerador da parcela anterior de 1, e � tal que o denominador � igual
ao produto do numerador por ele pr�prio mais 1.

Obs. 2: N�o esque�a de trocar o sinal de cada parcela para c�lculo do
somat�rio.

\item Escreva programa para ler repetidamente um n�mero inteiro
positivo $n$ e imprimir o seguinte, com $n$ linhas:

    \begin{center}
     \begin{tabular}{l}
      1\\
      2 3 \\
      3 4 5 \\
      4 5 6 7\\
      5 6 7 8 9\\ 
      \ldots
     \end{tabular}
     \end{center}

Ou seja, para cada $n$ lido, o programa deve imprimir, para
$i=1,\ldots,n$, uma linha com $i$ inteiros come�ando de $i$ e sendo
cada inteiro igual ao anterior na linha mais 1 e separado do anterior
por um espa�o.

O programa deve terminar quando o valor $n$ lido for menor ou igual a
zero.

\item Escreva um programa que leia palavra por palavra de um texto,
e imprima o total de: i) letras, ii) d�gitos e iii) caracteres de
pontua��o encontrados nessas palavras.

Cada palavra cont�m apenas letras, d�gitos ou caracteres de pontua��o
(como por exemplo v�rgula e ponto). 

Os caracteres delimitadores de palavras --- como {\tt '$\backslash$n'}
(terminador de linha), {\tt ' '} (espa�o) e {\tt '$\backslash$t'}
(tab) --- devem ser ignorados (isso � feito automaticamente pelo
m�todo {\it next\/} da classe {\it Scanner\/}).

O fim do texto deve ser testado usando o m�todo {\it hasNext\/} da
classe {\it Scanner\/} e cada palavra deve ser lida usando o m�todo
{\it next\/} dessa classe.

\end{enumerate}

\end{document}