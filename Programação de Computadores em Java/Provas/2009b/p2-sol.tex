\documentclass[brazil]{article}

\usepackage[latin1]{inputenc}
\usepackage[portuguese]{babel}
\usepackage{amsmath}
\usepackage{amssymb}

\title{Prova de Programa\c{c}\~ao de Computadores}

\author{Prof.: Carlos Camar\~ao}

\date{01 de Dezembro de 2009}

\setlength{\parindent}{0pt}

\pagestyle{empty}

\begin{document}

\maketitle

\begin{enumerate}

\item Escreva um programa que leia um n�mero inteiro positivo $n$, 
em seguida $n$ valores inteiros positivos $v_1,\ldots, v_n$, depois um
valor inteiros $m$ e um valor inteiro positivo $k$, e imprima todos os
valores dentre $v_1,\ldots, v_n$ que s�o maiores que $m$.

Os valores devem ser lidos do dispositivo de entrada padr�o e podem
estar separados entre si por um ou mais espa�os ou linhas.

A sa�da deve imprimir os valores separados por um espa�o, e contendo
no m�ximo $k$ valores por linha: a cada $k$ valores impressos deve ser
impresso um caractere de termina��o de linha.

Por exemplo, para a entrada:

{\tt \begin{tabbing}
10 1 3 5 7 9 6 7 8 9 10 5 2
\end{tabbing} }

a sa�da deve ser:

{\tt \begin{tabbing}
7 9\\
6 7\\
8 9\\
10
\end{tabbing} }

({\tt 7},{\tt 9},{\tt 6},{\tt 7},{\tt 8},{\tt 9},{\tt 10} s�o os
valores maiores que {\tt 5} dentre os {\tt 10} valores lidos).

{\bf Uma solu��o:}

\begin{verbatim}

\input{Q1.java}

\end{verbatim}


\item Escreva um programa que leia um valor inteiro positivo $n$,
em seguida $n$ valores inteiros positivos $v_1,v_2,\ldots, v_n$,
depois um valor positivo $m$, e imprima o valor resultante do c�lculo
de:

   \[ ((m-1) \times v_1) + ((m-2) \times v_2) + \ldots + ((m-n)\times v_n) \]

Os valores devem ser lidos do dispositivo de entrada padr�o e podem
estar separados entre si por um ou mais espa�os ou linhas.

Por exemplo, para a entrada:

{\tt \begin{tabbing}
4 1 2 3 4 10
\end{tabbing} }

a sa�da deve ser: {\tt 70} (pois esse � o resultado de $9\times 1 + 8
\times 2 + 7 \times 3 + 6 \times 4$).

\item Escreva um programa que leia um inteiro positivo $n$, depois
$n$ valores inteiros $a_1, \ldots, a_n$, e imprima uma mensagem
indicando se a sequ�ncia de valores $a_1, \ldots, a_n$ �
n�o-decrescente (ou seja, cada valor $a_{i+1}$ � maior ou igual a
$a_i$, para $i=1,\ldots,n-1$), n�o-crescente (ou seja, cada valor
$a_{i+1}$ � menor ou igual a $a_i$, para $i=1,\ldots,n-1$), constante
(todos os valores $a_i$ s�o iguais entre si, para $i=1,\ldots,n$), ou
nenhuma dessas op��es (nem n�o-decrescente, nem n�o-decrescente, nem
constante).

Por exemplo, para a entrada: 

{\tt \begin{tabbing}
4 1 2 3 3
\end{tabbing} }

a sa�da deve ser: {\tt \symbol{34}N�o-decrescente\symbol{34}} (pois a
sequ�ncia dos {\tt 4} valores {\tt 1},{\tt 2},{\tt 3},{\tt 3} �
n�o-decrescente).

Para a entrada: 

{\tt \begin{tabbing}
4 1 2 5 3
\end{tabbing} }

a sa�da deve ser: {\tt \symbol{34}Nem n�o-decrescente, nem
n�o-crescente, nem constante\symbol{34}} (pois a sequ�ncia dos {\tt 4}
valores {\tt 1},{\tt 2},{\tt 5},{\tt 3} n�o � n�o-decrescente, nem
n�o-decrescente, nem constante).

Dica: Use tr�s booleanos {\tt naoCrescente}, {\tt naoDecrescente},
{\tt constante}, inicialize-os com {\tt true} e atualize-os ao
percorrer (a partir do segundo valor da sequ�ncia) a sequ�ncia de
valores (armazenada em um arranjo) examinando valores adjacentes da
sequ�ncia.


\end{enumerate}

\end{document}