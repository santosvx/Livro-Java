\documentclass[brazil]{article}

\usepackage[latin1]{inputenc}
\usepackage[portuguese]{babel}
\usepackage{amsmath}
\usepackage{amssymb}

\title{Prova resgate}

\author{Prof.: Carlos Camar\~ao}

\date{15 de Dezembro de 2009}

\setlength{\parindent}{0pt}
\addtolength{\textwidth}{1cm}

\pagestyle{empty}

\begin{document}

\maketitle

\begin{enumerate}

\item Escreva um programa que leia um texto formado por uma seq��ncia de palavras, 
e imprima o n�mero de ocorr�ncias de cada letra e cada n�mero que
ocorre neste texto. Por exemplo, para a entrada: 

\begin{tabbing}
 xxxx\=\+\kill
 a bcd abcd efg xyyyy ....11122\\
\end{tabbing}

A sa�da deve ser:

\begin{tabbing}
  xxxx\=\+\kill
  a ocorre 1 vez\\
  b ocorre 2 vezes\\
  c ocorre 2 vezes\\
  d ocorre 1 vez\\
  e ocorre 1 vez\\
  f ocorre 1 vez\\
  g ocorre 1 vez\\
  x ocorre 1 vez\\
  y ocorre 4 vezes\\
  1 ocorre 3 vezes\\
  2 ocorre 2 vezes\\
\end{tabbing}


\item (10 pontos) Escreva em Java um programa que leia um valor
inteiro positivo {\tt numItens}, uma sequ�ncia de $n$ valores de tipo
{\tt double} que indicam o pre�o unit�rio desses itens, em seguida
v�rios pares {\tt ($k$,$v$)} de valores correspondentes a n�mero de um
item $k$ (de 1 a {\tt numItens}) e quantidades de itens vendidos em
uma opera��o de venda. 

Voc� pode considerar que os dados de entrada est�o corretos.

O programa deve imprimir, para cada item, o seu n�mero e o faturamento
obtido com vendas desse item.

\end{enumerate}

\end{document}

  

  


