\documentclass[brazil]{article}

\usepackage[latin1]{inputenc}
\usepackage[portuguese]{babel}
\usepackage{amsmath}
\usepackage{amssymb}

\title{Prova de Programa\c{c}\~ao de Computadores}

\author{Prof.: Carlos Camar\~ao}

\date{15 de Outubro de 2009}

\setlength{\parindent}{0pt}

\pagestyle{empty}

\begin{document}

\maketitle

\begin{enumerate}

\item Escreva um programa que leia repetidamente um valor inteiro
positivo, que expressa uma quantidade de segundos, e imprima, para
cada valor lido, uma mensagem que indica o n�mero de dias, horas,
minutos e segundos correspondente, como mostrado no exemplo a seguir.

Exemplo: para o valor lido $7322$, o programa deve imprimir a
mensagem:

  \[ \text{7322 segundos correspondem a 0 dias, 2 horas, 2 minutos e 2 segundos} \]

A entrada de dados e o programa devem terminar quando for lido um
valor menor ou igual a zero.

Voc� pode mas n�o precisa considerar que os substantivos dias, horas,
minutos e segundos devem estar no singular quando o valor
correspondente for igual a 1.

\item Escreva um programa que leia um inteiro positivo $n$ e imprima o
valor do somat�rio seguinte, com $n$ parcelas:

    \[ \frac{1}{2} - \frac{3}{5} + \frac{9}{10} - \frac{27}{17} +
\frac{81}{26} - \ldots \]

Obs. 1: N�o se esque�a: se o sinal de uma parcela � positivo, o sinal
da parcela seguinte � negativo e vice-versa.

Obs. 2: N�o use divis�o inteira (entre numerador e denominador de cada
parcela).

Obs.3: O numerador de uma parcela, seguinte a outra, pode ser obtido
multiplicando o numerador da parcela anterior por 3. O denominador de
uma parcela, seguinte a outra, pode ser obtido somando um certo valor
(que varia de parcela a parcela) ao denominador da parcela anterior.

\item Escreva programa para ler repetidamente um n�mero inteiro
positivo $n$ e imprimir o seguinte, com $n$ linhas:

    \begin{center}
     \begin{tabular}{l}
      1\\
      3 5 \\
      6 9 12 \\
      10 14 18 22\\
      15 20 25 30 35\\ 
      21 27 33 39 45 51\\
      \ldots
     \end{tabular}
     \end{center}

O programa deve terminar quando o valor $n$ lido for menor ou igual a
zero.

Dica 1: O primeiro n�mero de cada linha seguinte a outra pode ser
obtido a partir do primeiro n�mero da linha anterior somando um certo
valor.

Dica 2: O n�mero de valores em cada linha aumenta de 1 em 1.

Dica 3: O valor seguinte a outro em uma linha pode ser obtido a partir
do valor anterior somando um certo valor.


\end{enumerate}

\end{document}