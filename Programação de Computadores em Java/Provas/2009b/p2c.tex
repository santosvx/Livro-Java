\documentclass[brazil]{article}

\usepackage[latin1]{inputenc}
\usepackage[portuguese]{babel}
\usepackage{amsmath}
\usepackage{amssymb}

\title{Prova resgate}

\author{Prof.: Carlos Camar\~ao}

\date{15 de Dezembro de 2009}

\setlength{\parindent}{0pt}
\addtolength{\textwidth}{1cm}

\pagestyle{empty}

\begin{document}

\maketitle

\begin{enumerate}

\item Escreva um programa que leia um n�mero inteiro positivo $n$, 
em seguida $n$ valores inteiros $v_1,\ldots, v_n$, depois um n�mero
positivo $m$ e imprima: 

  \[ (v_1 + v_2 + \ldots + v_m) \times (v_{m+1} + v_{m+2} +  \ldots + v_{m+m}) \times \ldots \]
at� que se tenha $m$ ou menos valores restantes, e nesse �ltimo passo
deve-se multiplicar a soma desses $m$ ou menos valores
restantes.

Os valores devem ser lidos do dispositivo de entrada padr�o e podem
estar separados entre si por um ou mais espa�os ou linhas.

Por exemplo, para a entrada:

{\tt \begin{tabbing}
5 1 2 3 4 5 2
\end{tabbing} }

a sa�da deve ser: 105.

Isso ocorre porque $105 = (1+2) \times (3+4) \times 5$.

\item Escreva um programa que leia um texto em portugu�s, e imprima
uma lista que indique, para cada tamanho de palavra, o n�mero de vezes
que uma palavra desse tamanho ocorre ocorre no texto. 

Tamanhos que nunca ocorrem no texto n�o devem fazer parte da
lista. 

Voc� pode considerar que caracteres de pontua��o sempre ocorrem apenas
no final de uma palavra, e n�o devem ser considerados como parte da
palavra.

Por exemplo, para a entrada:

   \[ \text{Hoje estou fazendo uma prova de PC; est� legal.} \]

A sa�da deve ser: 

\begin{tabbing}
   xxxxx\=\+\kill
   Tamanho 2: 2 ocorr�ncias\\
   Tamanho 3: 1 ocorr�ncia\\
   Tamanho 4: 2 ocorr�ncias\\
   Tamanho 5: 3 ocorr�ncias\\
   Tamanho 7: 1 ocorr�ncia
\end{tabbing}

No exemplo, temos:

\begin{tabbing}
  xxxxx\=\+\kill
    As ocorr�ncias de palavras de tamanho 2 s�o: {\tt de}, {\tt PC} \\
    A �nica ocorr�ncia de palavra de tamanho 3 �: {\tt uma} \\
    As ocorr�ncias de palavras de tamanho 4 s�o: {\tt Hoje}, {\tt est�} \\
    As ocorr�ncias de palavras de tamanho 5 s�o: {\tt estou}, {\tt prova}, {\tt legal}\\
    A �nica ocorr�ncia de palavra de tamanho 7 �: {\tt fazendo}
\end{tabbing}

O texto deve ser lido do dispositivo de entrada padr�o, usando o
m�todo {\tt next} da classe {\tt Scanner} para ler uma palavra
(considere que {\tt next} retorna a pr�xima palavra do texto,
incluindo poss�vel caractere de pontua��o no final da palavra, e
avan�a a posi��o de leitura para a pr�xima palavra). Por exemplo, ao
chamar o m�todo {\tt next} para ler {\tt \symbol{34}PC;\symbol{34}}, a
cadeia de caracteres {\tt \symbol{34}PC;\symbol{34}} � retornada,
incluindo o caractere {\tt \symbol{34};\symbol{34}} como �ltimo
caractere da cadeia.

O fim do texto deve ser testado por meio do uso do m�todo {\tt
hasNext} da classe {\tt Scanner}.

Considere que palavras t�m no m�ximo 30 caracteres. 

\end{enumerate}

\end{document}