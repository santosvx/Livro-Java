\documentclass[brazil]{article}

\usepackage[latin1]{inputenc}
\usepackage[portuguese]{babel}
\usepackage{amsmath}
\usepackage{amssymb}

\title{Prova resgate}

\author{Prof.: Carlos Camar\~ao}

\date{15 de Dezembro de 2009}

\setlength{\parindent}{0pt}
\addtolength{\textwidth}{1cm}

\pagestyle{empty}

\begin{document}

\maketitle

\begin{enumerate}

\item Escreva um programa que leia um n�mero inteiro positivo $n$, 
em seguida $n$ valores inteiros $v_1,\ldots, v_n$, depois um n�mero
positivo $m$ e imprima: 

  \[ (v_1\times v_2\times v_m) + (v_{m+1}\times v_{m+2} \times \ldots \times v_{m+m}) + \ldots \]
at� que se tenha $m$ ou menos valores restantes, e nesse �ltimo passo
deve-se somar a multiplica��o desses $m$ ou menos valores
restantes.

Os valores devem ser lidos do dispositivo de entrada padr�o e podem
estar separados entre si por um ou mais espa�os ou linhas.

Por exemplo, para a entrada:

{\tt \begin{tabbing}
5 1 2 3 4 5 2
\end{tabbing} }

a sa�da deve ser: 19

Isso ocorre porque $19 = (1\times 2) + (3\times 4) + 5$.

\item Escreva um programa para determinar o vencedor de uma elei��o,
dados o n�mero de candidatos e os votos.

O primeiro valor da entrada � um inteiro $n$ representando o n�mero de
candidatos. Cada valor seguinte � um inteiro positivo entre 1 e $n$
que representa um voto (os candidatos s�o identificados por inteiros
positivos entre 1 e $n$).

O final dos dados � indicado por um voto igual a zero. Voc� pode supor
que os dados est�o corretos (o n�mero de candidatos e cada voto s�o
inteiros positivos). Cada inteiro da entrada � separado do seguinte
por um ou mais espa�os ou linhas.

A sa�da deve conter o n�mero do candidato que venceu (aquele que
obteve mais votos). Voc� pode supor que existe apenas um vencedor.

Exemplo: para a entrada:

{\tt 
\begin{tabular}{l}
4 1 2 1 4 3 0
\end{tabular} }

a sa�da deve ser:

{\tt 1}

Isso ocorre porque o candidato {\tt 1} foi o candidato que recebeu
mais votos (2 votos).

\end{enumerate}

\end{document}