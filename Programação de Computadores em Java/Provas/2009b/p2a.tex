\documentclass[brazil]{article}

\usepackage[latin1]{inputenc}
\usepackage[portuguese]{babel}
\usepackage{amsmath}
\usepackage{amssymb}

\title{Prova de Programa\c{c}\~ao de Computadores}

\author{Prof.: Carlos Camar\~ao}

\date{01 de Dezembro de 2009}

\setlength{\parindent}{0pt}

\pagestyle{empty}

\begin{document}

\maketitle

\begin{enumerate}

\item Escreva um programa que leia um n�mero inteiro positivo $n$, 
em seguida $n$ valores inteiros positivos $v_1,\ldots, v_n$, depois um
valor inteiro $m$, e imprima a soma dos valores dentre $v_1,\ldots,
v_n$ que s�o maiores que $m$.

Os valores devem ser lidos do dispositivo de entrada padr�o e podem
estar separados entre si por um ou mais espa�os ou linhas.

Por exemplo, para a entrada:

{\tt \begin{tabbing}
4 1 2 3 4 2
\end{tabbing} }

a sa�da deve ser: {\tt 7} (pois {\tt 3} e {\tt 4} s�o os valores
maiores que {\tt 2} dentre os {\tt 4} valores {\tt 1}, {\tt 2}, {\tt
3}, {\tt 4}).

\item Escreva um programa que leia um valor inteiro positivo $n$,
em seguida $n$ valores inteiros positivos $v_1,v_2,\ldots, v_n$,
depois um valor positivo $m$, e imprima o valor resultante do c�lculo
de:

   \[ ((m-1) + v_1) \times ((m-2) + v_2) \times \ldots \times ((m-n) + v_n) \]

Os valores devem ser lidos do dispositivo de entrada padr�o e podem
estar separados entre si por um ou mais espa�os ou linhas.

Por exemplo, para a entrada:

{\tt \begin{tabbing}
4 1 2 3 4 10
\end{tabbing} }

a sa�da deve ser: {\tt 1000} (pois esse � o resultado de $(9+1) \times
(8+2) \times (7+3) \times (6+4)$).

\item Escreva um programa que leia um inteiro positivo $n$, depois
$n$ valores inteiros $a_1, \ldots, a_n$, e imprima uma mensagem
indicando se a sequ�ncia de valores $a_1, \ldots, a_n$ � crescente (ou
seja, cada valor $a_{i+1}$ � maior que $a_i$, para $i=1,\ldots,n-1$),
decrescente (ou seja, cada valor $a_{i+1}$ � menor que $a_i$, para
$i=1,\ldots,n-1$), constante (todos os valores $a_i$ s�o iguais entre
si, para $i=1,\ldots,n$), ou nenhuma dessas op��es (nem crescente, nem
decrescente, nem constante).

Por exemplo, para a entrada: 

{\tt \begin{tabbing}
4 1 2 3 4
\end{tabbing} }

a sa�da deve ser: {\tt \symbol{34}Crescente\symbol{34}} (pois a
sequ�ncia dos {\tt 4} valores {\tt 1},{\tt 2},{\tt 3},{\tt 4} �
crescente).

Para a entrada: 

{\tt \begin{tabbing}
4 1 2 3 3
\end{tabbing} }

a sa�da deve ser: {\tt \symbol{34}Nem crescente, nem decrescente, nem
constante\symbol{34}} (pois a sequ�ncia dos {\tt 4} valores {\tt
1},{\tt 2},{\tt 3},{\tt 3} n�o � crescente, nem decrescente, nem
constante).

Dica: Use tr�s booleanos {\tt crescente}, {\tt decrescente}, {\tt
constante}, inicialize-os com {\tt true} e atualize-os ao percorrer (a
partir do segundo valor da sequ�ncia) a sequ�ncia de valores
(armazenada em um arranjo) examinando valores adjacentes da sequ�ncia.

\end{enumerate}

\end{document}