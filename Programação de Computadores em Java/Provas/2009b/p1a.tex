\documentclass[brazil]{article}

\usepackage[latin1]{inputenc}
\usepackage[portuguese]{babel}
\usepackage{amsmath}
\usepackage{amssymb}

\title{Prova de Programa\c{c}\~ao de Computadores}

\author{Prof.: Carlos Camar\~ao}

\date{15 de Outubro de 2009}

\setlength{\parindent}{0pt}

\pagestyle{empty}

\begin{document}

\maketitle

\begin{enumerate}

\item Escreva um programa que leia, repetidamente, do dispositivo de
entrada padr�o, valores, separados por um ou mais espa�os ou linhas,
de CPF e renda mensal, em reais, de contribuintes e imprima, para cada
contribuinte (i.e.~cada cpf), o imposto de renda a ser pago, calculado
conforme a seguinte regra:
  
  \[ \text{\begin{tabular}{ll}
       Renda mensal at� R\$1164,00 & Isento\\
       Renda mensal maior que R\$1164,00 e menor que R\$2326,00 & 15\% de imposto de renda\\
       Renda mensal superior a R\$2326,00 & 27,5\% de imposto de renda
     \end{tabular} }
   \]

A impress�o deve conter cada CPF, um espa�o, e o imposto de renda a
ser pago.

A leitura deve terminar quando um valor de CPF menor ou igual a zero
for lido.

Considere que os dados de entrada sempre est�o corretos e que cada
valor de renda mensal consiste em um valor v�lido de tipo {\tt float}.

\item Escreva um programa que leia um inteiro positivo $n$ e imprima o
valor do somat�rio seguinte, com $n$ parcelas:

    \[ \frac{1}{2} - \frac{4}{3} + \frac{16}{5} - \frac{64}{8} +
\frac{256}{12} - \ldots \]

Obs. 1: N�o se esque�a: se o sinal de uma parcela � positivo, o sinal
da parcela seguinte � negativo e vice-versa.

Obs. 2: N�o use divis�o inteira (entre numerador e denominador de cada
parcela).

Obs.3: O numerador de uma parcela, seguinte a outra, pode ser obtido
multiplicando o numerador da parcela anterior por 4. O denominador de
uma parcela, seguinte a outra, pode ser obtido somando um certo valor
(que varia de parcela a parcela) ao denominador da parcela anterior.

\item Escreva programa para ler repetidamente um n�mero inteiro
positivo $n$ e imprimir o seguinte, com $n$ linhas:

    \begin{center}
     \begin{tabular}{l}
      1\\
      3 4 \\
      6 8 10 \\
      10 13 16 19\\
      15 19 23 27 31\\ 
      21 26 31 36 41 46\\
      \ldots
     \end{tabular}
     \end{center}

O programa deve terminar quando o valor $n$ lido for menor ou igual a
zero.

Dica 1: O primeiro n�mero de cada linha seguinte a outra pode ser
obtido a partir do primeiro n�mero da linha anterior somando um certo
valor.

Dica 2: O n�mero de valores em cada linha aumenta de 1 em 1.

Dica 3: O valor seguinte a outro em uma linha pode ser obtido a partir
do valor anterior somando um certo valor.

\end{enumerate}

\end{document}