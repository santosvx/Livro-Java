\documentclass{article}

\title{1$^{\underline{{\rm{a}}}}$ Prova de Programa\c{c}\~ao de Computadores}

\date{4 de Dezembro de 2003}

\author{Valor: 20 pontos}

\begin{document}

\maketitle

\begin{enumerate}

\item (5 pontos) Escreva em Java uma fun\c{c}\~ao (m\'etodo
est\'atico) que receba como par\^ametro um n\'umero inteiro
n\~ao-negativo $n$ de at\'e tr\^es d\'{\i}gitos e retorne como
resultado o valor da multiplica\c{c}\~ao de $sd$ por $md$, onde $sd$
\'e a soma dos d\'{\i}gitos de $n$ e $md$ \'e igual ao maior
d\'{\i}gito de $n$.

Fa\c{c}a um programa que leia (usando {\tt {\it JOptionPane}.{\it
showInputDialog\/}}) um valor inteiro, repetidamente, at\'e que o
valor lido n\~ao seja um valor entre 0 e 999, e chame a fun\c{c}\~ao
acima para cada um dos valores lidos, imprimindo o resultado calculado
pela fun\c{c}\~ao.

\item (5 pontos) Reescreva a fun\c{c}\~ao da quest\~ao anterior de
forma a que ela n\~ao suponha que o argumento tenha at\'e tr\^es
d\'{\i}gitos, ou seja, a sua fun\c{c}\~ao deve funcionar para qualquer
valor positivo de tipo {\tt int}.

Fa\c{c}a um programa que leia (usando {\tt {\it JOptionPane}.{\it
showInputDialog\/}}) um valor inteiro, repetidamente, at\'e que o valor
lido seja um valor negativo, e chame a fun\c{c}\~ao acima para cada um
dos valores lidos, imprimindo o resultado calculado pela fun\c{c}\~ao.

\item (5 pontos) Escreva em Java uma fun\c{c}\~ao (m\'etodo
est\'atico) que receba como par\^ametro um n\'umero inteiro positivo
$n$ e retorne o valor da soma das $n$ primeiras parcelas do
somat\'orio \[
\frac{1}{1} - \frac{2}{4} + \frac{3}{9} - \frac{4}{16} +
\frac{5}{25} - \frac{6}{36} + \ldots \]

\item (5 pontos) Fa\c{c}a um programa para imprimir a seguinte tabela:

 \[ \begin{array}{llllllllll}
      1 & 2 & 3 & 4 & 5 & 6 & 7 & 8 & 9 & 10\\
      2 & 4 & 6 & 8 & 10 & 12 & 14 & 16 & 18 & 20\\
      \ldots\\
      10 & 20 & 30 & 40 & 50 & 60 & 70 & 80 & 90 & 100
    \end{array}
  \]
Use espa\c{c}os para controlar o formato da tabela.

\end{enumerate}

\end{document}