\documentclass[brazil]{article}

\title{Prova de Programa\c{c}\~ao de Computadores}

\author{Prof.: Carlos Camar\~ao}

\date{24 de Julho de 2003}

\addtolength\textwidth{15mm}
\setlength{\parindent}{0pt}

\pagestyle{empty}

\begin{document}

\maketitle

\newcommand{\prog}[1]{{\tt \begin{tabbing}
                        #1 \end{tabbing} }}
\begin{enumerate}

\item (15 pontos)

\begin{enumerate}

 \item ({\it 5 pontos\/}) Escreva em Java um m\'etodo est\'atico que
 leia uma cadeia de ca\-rac\-teres e converta essa cadeia de
 caracteres em um inteiro.  Se a cadeia de caracteres n\~ao
 representar um valor inteiro, a exce\c{c}\~ao {\it
 NumberFormatException\/} deve ser propagada pelo m\'etodo ({\em
 dica\/}: use o m\'etodo {\it parseInt\/} da classe {\it Integer\/},
 que causa essa exce\c{c}\~ao quando o objeto de tipo {\it String\/}
 passado como argumento n\~ao representa um valor inteiro). Caso
 contr\'ario, o valor inteiro lido deve ser retornado (como um valor
 do tipo {\tt int}).

 \item ({\it 5 pontos\/}) Usando o m\'etodo est\'atico definido acima,
 escreva um programa que leia um valor inteiro positivo \'{\i}mpar, e
 imprima um losango como o mostrado abaixo, onde o n\'umero de
 caracteres {\tt *} na linha do meio do losango \'e igual ao valor
 lido.

 Se o n\'umero lido n\~ao for um valor inteiro ou n\~ao for um valor
 inteiro positivo \'{\i}mpar, uma exce\c{c}\~ao deve ser causada e
 tratada, emitindo-se uma mensagem apropriada --- respectivamente {\em
 ``Era esperado um valor inteiro\/} e {\em ``Era espe\-ra\-do um valor
 inteiro positivo {\'{\i}mpar\/}''\/} ---, e a execu\c{c}\~ao do
 programa deve ser terminada.

 \item ({\it 5 pontos\/}) Modifique o programa (mas n\~ao o m\'etodo
 est\'atico) anterior para que o programa agora n\~ao termine a
 execu\c{c}\~ao quando um valor n\~ao esperado for lido, mas leia
 outro valor, at\'e que o valor digitado seja um valor v\'alido
 (inteiro positivo \'{\i}mpar), quando s\'o ent\~ao o losango descrito
 deve ser impresso e a execu\c{c}\~ao do programa deve terminar.

\end{enumerate}

\begin{center}
\begin{tabular}{c}
\ *\ \\
\ ***\ \\
\ *****\ \\
\ *******\ \\
\ *****\ \\
\ ***\ \\
\ *\ \\
\end{tabular}
\end{center}



\item ({\em 5 pontos\/}) Considere o trecho de program a seguir:

  \prog{class {\it ArranjoDeInteiros\/}\\ \{ \=int[] {\it a\/};\+\\ \\
          
          {\it ArranjoDeInteiros\/} (int $n$)
          \{ $a$ = new int[$n$]; \}\\ \\

          boolean {\it eq\/} ({\it ArranjoDeInteiros b\/}) \{ \ldots\ \}\-\\
        \} 
       }

Escreva uma implementa\c{c}\~ao para o m\'etodo {\it eq\/}, de tal que
forma que, (para quaisquer {\it a1\/} e {\it a2\/} do tipo {\it
ArranjoDeInteiros\/}, {\tt {\it a1}.{\it eq\/}({\it a2\/})} compare a
igualdade dos valores armazenados no arranjo {\it a\/} de {\it a1\/}
com os valores armazenados no arranjo $a$ de {\it a2\/} e retorne {\tt
true} se esses valores s\~ao todos iguais e {\tt false} em caso
contr\'ario. N\~ao esque\c{c}a de considerar os tamanhos dos arranjos
(i.e.~dois arranjos s\~ao diferentes se t\^em tamanhos diferentes).

\item ({\it 5 pontos\/}) Um armaz\'em trabalha com $n$ mercadorias
diferentes ($n=100$), identificadas por n\'umeros de 1 a $n$.

Um funcion\'ario desse armaz\'em tem sal\'ario mensal estipulado como
20\% da receita mensal obtida com as vendas que ele realizou, mais um
b\^onus igual a 10\% da receita mensal obtida com o produto que teve o
maior n\'umero de unidades vendidas.

Escreva um programa que leia, para cada mercadoria de 1 a $n$, o
n\'umero de unidades vendidas em um determinado m\^es por esse
funcion\'ario e o pre\c{c}o unit\'ario da mercadoria, e imprima o
sal\'ario do funcion\'ario, nesse m\^es.

Para leitura dos dados, voc\^e pode supor que existam m\'etodos
est\'aticos {\it readInt\/} e {\it readFloat\/}, que n\~ao t\^em
par\^ametros e retornam respectivamente um valor do tipo {\tt int} e
um valor do tipo {\tt float}, lidos do dispositivo de entrada
padr\~ao. Suponha que esses m\'etodos s\~ao definidos em uma classe
chamada (digamos) {\it Console\/}.

{\em Dica\/}: na sua implementa\c{c}\~ao, defina uma classe com:

\begin{itemize}

  \item uma vari\'avel inteira (que representa o valor de $n$), um
  arranjo de inteiros (que ir\'a armazenar para cada mercadoria o
  n\'umero de unidades vendidas) e um arranjo de pre\c{c}os (que ir\'a
  armazenar para cada mercadoria o seu pre\c{c}o unit\'ario).

  \item m\'etodo para leitura dos dados e para c\'alculo do sal\'ario
  do funcion\'ario do armaz\'em.

\end{itemize}

\end{enumerate} 

\end{document}