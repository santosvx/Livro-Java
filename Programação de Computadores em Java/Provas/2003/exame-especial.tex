\documentstyle{article}

\title{Exampe Especial\\ Algoritmos e Estruturas de Dados I}

\pagestyle{empty}	

\author{Prof.: Carlos Camar\~ao}
\date{4 de Setembro de 2003}

\begin{document}

\maketitle
\thispagestyle{empty}

\begin{enumerate}

\item Escreva um programa que leia um valor inteiro positivo $n$, 
calcule o valor de $\sum_{i=1}^n (-i/(i*i))$, e imprima esse valor.
Decomponha o seu programa em tr\^es partes, cada uma delas
implementada por um m\'etodo: leitura, c\'alculo do valor desejado a a
partir do valor lido, e impress\~ao do valor calculado.

O programa (isto \'e, o m\'etodo de leitura) deve ler uma cadeia de
caracteres e causar uma exce\c{c}\~ao se essa cadeia n\~ao representar
um n\'umero inteiro positivo. Nesse caso, a exce\c{c}\~ao deve ser
tratada no programa principal, e esse tratamento deve apenas emitir
uma mensagem de erro apropriada, e em seguida a execu\c{c}\~ao do
programa deve ser terminada.

\item Fa\c{c}a um programa para ler um valor inteiro positivo $n$, e
em seguida $n$ seq\"u\^encias de valores inteiros positivos, cada uma
delas terminada por um valor inteiro negativo ou nulo, e imprimir a
soma dos maiores valores de cada uma das seq\"u\^encias. 

Por exemplo, se os valores digitados pelo usu\'ario forem:
  
  \[ \begin{array}{l}
        3    \\
        1    \\
        10   \\
        0    \\
        2    \\
        20   \\
        10   \\
        -1   \\
        3    \\ 
        30   \\
        5    \\
        1    \\
        0    
      \end{array} \]
o programa deve imprimir 50 como resultado (50 = 10+20+30).

Em cada situa\c{c}\~ao de entrada de dados, se um valor n\~ao esperado
for fornecido com entrada (por exemplo uma cadeia de caracteres que
n\~ao representa um n\'umero inteiro), uma exce\c{c}\~ao deve ser
causada, e o processo de entrada de dados pelo usu\'ario do programa
deve recome\c{c}ar, a partir do in\'{\i}cio (isto \'e, a partir da
leitura de $n$). 

Dica 2: o m\'etodo {\it parseInt\/} causa uma exce\c{c}\~ao ({\it
NumberFormatException\/}) se a cadeia de caracteres fornecida como
argumento n\~ao representar um n\'umero inteiro.

Dica 3: N\~ao esque\c{c}a de testar se o primeiro valor lido (de $n$)
\'e um valor positivo, e causar uma exce\c{c}\~ao em caso contr\'ario. 
Voc\^e pode causar e tratar apenas a exce\c{c}\~ao {\it Exception\/},
pois a exce\c{c}\~ao {\it NumberFormatException\/} \'e um objeto da
classe {\it Exception\/}, e portanto ser\'a tamb\'em tratada por um
tratador que trata (recebe como argumento) uma exce\c{c}\~ao {\it
NumberFormatException\/}.

\item Escreva outra vers\~ao para o seu programa acima, que difere da
anterior da seguinte maneira. Se o seu programa declarou e usou um
arranjo, fa\c{c}a uma vers\~ao que n\~ao declara e nem usa um
arranjo. E vice-versa (se o seu programa n\~ao declarou e usou um
arranjo, fa\c{c}a uma vers\~ao que declara e usa um arranjo).

\item Fa\c{c}a um programa que leia dois n\'umeros inteiros positivos,
$n$ e $m$, e $n$ seq\"u\^encias de $m$ valores inteiros positivos,
cada uma dessas seq\"u\^encias terminada por um valor inteiro negativo
ou zero, e imprima, para $i$ variando de 1 a $m$, a soma dos
$i$-\'esimos inteiros de cada uma das $n$ sequ\^encias. 

Por exemplo, se os valores fornecidos como entrada forem:
  
  \[ \begin{array}{l}
        3    \\
        2    \\
        10   \\
        5    \\
        0    \\
        2    \\
        20   \\
        -1   \\
        3    \\
        30   \\
        0
      \end{array} \]
o programa deve imprimir 70 como resultado: 
  70 = (10+2+3) + (5+20+30) 

Se um valor n\~ao esperado for fornecido com entrada (por exemplo uma
cadeia de caracteres que n\~ao representa um n\'umero inteiro ou se o
n\'umero de inteiros positivos de cada seq\"u\^encia n\~ao for o
mesmo), uma exce\c{c}\~ao deve ser causada, e o processo de entrada de
dados pelo usu\'ario do programa deve recome\c{c}ar, a partir do
in\'{\i}cio.

\end{enumerate}

\end{document}

