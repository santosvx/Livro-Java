\documentclass[brazil]{article}

\usepackage[latin1]{inputenc}

\title{Prova de Programa\c{c}\~ao de Computadores}

\author{Prof.: Carlos Camar\~ao}

\date{28 de Abril de 2005}

\addtolength\textwidth{15mm}
\setlength{\parindent}{0pt}

\pagestyle{empty}

\begin{document}

\maketitle

Em todas as quest\~oes, voc\^e pode supor a exist\^encia de m\'etodos
est\'aticos de entrada de valores de tipos b�sicos, definidos em uma
classe {\tt EntradaPadr\~ao}. Por exemplo, para ler inteiros, voc\^e
pode usar {\tt EntradaPadr\~ao.l\^eInteiro()}. Esse m�todo l� um
inteiro do dispositivo de entrada padr�o e retorna o valor lido, de
tipo {\tt int}. 

\begin{enumerate}

\item (6 pontos) Escreva em Java um programa que leia dois valores inteiros
positivos $n$ e $m$, que representam respectivamente o n�mero de
quest�es de uma prova de m�ltipla escolha e o n�mero de alunos de uma
turma. Em seguida, o programa deve ler, para cada um dos $m$ alunos,
$n$ valores inteiros n�o-negativos, que representam as respostas de
cada aluno para cada quest�o da prova. Depois disso, o programa deve
ler, para cada uma das quest�es, um valor inteiro que indica a
resposta correta de cada quest�o. O programa deve imprimir, para cada
aluno, o seu n�mero (de 1 a $m$), e o n�mero de respostas corretas
dadas por esse aluno. 

\item (7 pontos) Madame Zen \'e uma figura m\'{\i}stica que, al\'em de
prever o futuro lendo a m\~ao das pessoas, acredita que a frequ\^encia
de ocorr\^encia dos d\'{\i}igitos na representa\c{c}\~ao decimal dos
fatoriais de n\'umeros naturais est\'a relacionada ao futuro das
pessoas. Ela decidiu vender seus servi\c{c}os, fazendo previs\~oes
sobre a vida de seus fregueses, baseadas nessas freq\"u\^encias, e
pediu voc\^e para ajud\'a-la, determinando esses valores para ela.

Ela espera usar dados de entrada como os dias da semana, do m\^es e do
ano como valores de entrada, que s\~ao valores positivos menores ou
iguais a 365, e voc\^e deve ent\~ao fazer um programa que leia uma
lista de n\'umeros naturais e imprima, para cada um, o n\'umero de
ocorr\^encias de cada d\'{\i}gito decimal na representa\c{c}\~ao
decimal do fatorial desse n\'umero. O programa deve terminar quando o
n\'umero lido for maior do que 365 ou menor ou igual a zero.

Por exemplo, para a entrada:

\begin{tabular}{l}
3\\
8\\
100\\
0
\end{tabular}

o programa deve imprimir uma tabela como a seguinte:

\begin{tabular}{ll}
3!   & \\
     & (0)    0    (1)    0    (2)    0    (3)    0    (4)    0\\
     & (5)    0    (6)    1    (7)    0    (8)    0    (9)    0\\
8!   & \\
     & (0)    2    (1)    0    (2)    1    (3)    1    (4)    1\\
     & (5)    0    (6)    0    (7)    0    (8)    0    (9)    0\\
100! & \\
     & (0)   30    (1)   15    (2)   19    (3)   10    (4)   10\\
     & (5)   14    (6)   19    (7)    7    (8)   14    (9)   20
\end{tabular}

Os dados devem ser lidos do dispositivo de entrada padr\~ao.


\item (7 pontos) Um numeral \'e um {\em pal\'{\i}ndromo\/} se ele \'e  
o mesmo se lido da esquerda para a direita ou da direita para a
esquerda (em outras palavras, se ele \'e igual ao seu reverso). Por
exemplo, 74547 \' um pal\'{\i}ndromo.

Um n\'umero pode ter uma representa\c{c}\~ao que \'e um
pal\'{\i}ndromo em uma base, e n\~ao \'e em outra. Por exemplo, 17
n\~ao \'e um pal\'{\i}ndromo, mas 10001 \'e, e ambas representam o
mesmo n\'umero se as bases s\~ao 10 e 2, respectivamente.

Fa\c{c}a um programa que leia uma lista de numerais na base 10 e
imprima, para cada $i$ dessa lista, uma das seguintes mensagens,
conforme apropriado:

  \begin{itemize}
 
     \item ``A representa\c{c}\~ao de $i$ \'e um pal\'{\i}ndromo nas
     bases: $b_1, \ldots, b_n$'' (onde $b_1, \ldots, b_n$ s\~ao as
     bases de 2 a 16 nas quais a representa\c{c}\~ao de $i$ \'e um
     pal\'{\i}ndromo);

     \item ``A representa\c{c}\~ao de $i$ n\~ao \'e um pal\'{\i}ndromo
     em nenhuma das bases de 2 a 16''.

  \end{itemize}

A leitura dos numerais deve terminar quando o valor lido for igual ou
menor do que zero.

\end{enumerate}

\end{document}