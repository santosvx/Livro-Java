\documentclass[brazil]{article}

\usepackage[latin1]{inputenc}

\title{Prova de Programa\c{c}\~ao de Computadores}

\author{Prof.: Carlos Camar\~ao}

\date{28 de Abril de 2005}

\addtolength\textwidth{15mm}
\setlength{\parindent}{0pt}

\pagestyle{empty}

\begin{document}

\maketitle

\begin{enumerate}

\item (6 pontos) Escreva em Java uma fun\c{c}\~ao (m\'etodo
est\'atico) que, dado um valor $n$ n\~ao-negativo, calcule o
somat\'orio dos $n$ primeiros termos da seq\"u\^encia $1, - 2, 4, - 8,
16, - 32, \ldots$ Se $n$ for igual a 0 (zero), o resultado do
somat\'orio \'e igual a 0 (zero).

Fa\c{c}a um programa que leia repetidamente um valor inteiro, at\'e
que o valor lido seja menor que zero, e chame a fun\c{c}\~ao acima
para cada um dos valores positivos lidos, imprimindo o resultado
calculado pela fun\c{c}\~ao.

\item (7 pontos) Escreva uma fun\c{c}\~ao (m\'etodo est\'atico) em Java que
receba um n\'umero inteiro posi\-tivo $n$ como par\^ametro e retorne o
valor de $sd(n) * md(n)$, onde $sd(n)$ � a soma dos d\'{\i}gitos de
$n$ e $md(n)$ o maior d�gito (da representa��o decimal) de $n$.

Fa\c{c}a um programa que leia repetidamente um valor inteiro, at\'e
que o valor lido seja menor ou igual a zero, e chame a fun\c{c}\~ao
acima para cada um dos valores positivos lidos, imprimindo o resultado
calculado pela fun\c{c}\~ao.

\item (7 pontos) Suponha que, sob determinadas condi\c{c}\~oes, a
radioatividade de um determinado material diminui exponencialmente, de
forma que, a cada instante $t+1$, sua radioatividade passe a ser a
metade de sua radoatividade no instante $t$. A radioatividade \'e
medida em alguma unidade de radioatividade (qual, especificamente,
n\~ao \'e importante para solu\c{c}\~ao da quest\~ao). A unidade de
tempo tamb�m n�o � importante para solu��o da quest�o. Defina uma
fun\c{c}\~ao (m\'etodo est\'atico) em Java com dois par\^ametros $r1$
e $r2$, que representam medidas de radio\-a\-tividade, e calcule por
quantas unidades de tempo esse material, nessas condi\c{c}\~oes,
iniciando com radioatividade igual a $r1$, ainda tem uma
radioatividade maior que $r2$ (esse tempo
\'e igual a zero se $r1 \leq r2$).

Fa\c{c}a um programa que leia repetidamente dois n\'umeros de ponto
flutuante, um em seguida do outro, at\'e que pelo menos um deles seja
menor ou igual a zero, e chame a fun\c{c}\~ao acima para cada par de
valores positivos lidos, imprimindo o resultado calculado pela
fun\c{c}\~ao. Em cada caso, o primeiro valor lido deve definir o valor
de $r1$ (radioatividade inicial do material) e o segundo valor lido
deve definir o valor de $r2$ (limite m\'{\i}nimo de radioatividade).

\end{enumerate}

\end{document}