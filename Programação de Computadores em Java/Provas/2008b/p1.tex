\documentclass[brazil]{article}

\usepackage[latin1]{inputenc}
\usepackage[portuguese]{babel}

\title{Prova de Programa\c{c}\~ao de Computadores}

\author{Prof.: Carlos Camar\~ao}

\date{10 de Novembro de 2008}

\addtolength\textwidth{15mm}
\setlength{\parindent}{0pt}

\pagestyle{empty}

\begin{document}

\maketitle

\begin{enumerate}

\item Escreva um programa que leia, repetidamente, valores inteiros
positivos e imprima, para cada inteiro $n$ lido, o valor m\'edio dos
divisores {\em pr\'oprios\/} de $n$ --- calculada como a soma dos
divisores pr\'oprios de $n$ dividida pelo n\'umero de divisores
pr\'oprios de $n$).

A leitura deve terminar quando um valor negativo ou zero for lido.

Um divisor pr\'oprio de $n$ \'e um divisor de $n$ diferente de $1$ e
de $n$.

A soma e o n\'umero de divisores devem ser inteiros. O resultado (a
m\'edia) deve ser um n\'umero de ponto flutuante. N\~ao use divis\~ao
entre dois inteiros para c\'alculo da m\'edia; converta a soma ou o
n\'umero de divisores de modo que a divis\~ao feita para c\'alculo da
m\'edia seja uma divis\~ao de n\'umeros de ponto flutuante.

\item Escreva um programa que leia repetidamente um valor inteiro,
at\'e que o valor lido seja menor que zero, e chame o m\'etodo
est\'atico {\it somaPot2PosNeg\/}, descrito abaixo, para cada um dos
valores positivos lidos, imprimindo o resultado calculado pelo
m\'etodo.

O m\'etodo est\'atico {\it somaPot2PosNeg\/} calcula, para um dado um
valor $n$ n\~ao-negativo, o somat\'orio dos $n$ primeiros termos da
seq\"u\^encia $1, - 2, 4, - 8, 16, - 32, \ldots$ Se $n$ for igual a
zero, o resultado do somat\'orio \'e igual a zero.

\item Escreva um programa para ler um n\'umero positivo $n$ e imprimir
a soma de todos os m\'ultiplos de 3 ou 5 que s\~ao menores que $n$.

Por exemplo, se for lido 10 o programa deve imprimir 23 --- uma vez
que a soma de todos os m\'ultiplos de 3 ou 5 que s\~ao menores que 10
\'e igual a (3+5+6+9)=23.


\end{enumerate}

\end{document}