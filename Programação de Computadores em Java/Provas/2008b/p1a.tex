\documentclass[brazil]{article}

\usepackage[latin1]{inputenc}
\usepackage[portuguese]{babel}

\title{Prova de Programa\c{c}\~ao de Computadores}

\author{Prof.: Carlos Camar\~ao}

\date{11 de Novembro de 2008}

\addtolength\textwidth{15mm}
\setlength{\parindent}{0pt}

\pagestyle{empty}

\begin{document}

\maketitle

\begin{enumerate}

\item Escreva um programa que leia um texto --- um valor de tipo {\it
String\/} --- e imprima o n\'umero de caracteres e o n\'umero de
linhas deste texto. O n\'umero de linhas \'e igual ao n\'umero de
caracteres {\tt '$\backslash$n'} presentes no texto.

\item Escreva um programa que leia, repetidamente, valores inteiros,
at\'e que um valor negativo ou zero seja lido, e imprima, para cada
valor positivo $n$ lido, uma seq\"u\^encia de $n$ linhas, tais que a
$i$-\'esima linha cont\'em $i$ caracteres {\tt '*'}.

\item Escreva um programa que leia v\'arios {\em textos\/} --- valores
de tipo {\it String\/} --- e imprima, para cada texto lido, se ele
cont\'em alguma {\em palavra\/} pal\'{\i}ndrome e, em caso positivo, a
primeira palavra pal\'{\i}ndrome encontrada.

Considere uma palavra qualquer valor de tipo {\it String\/} retornado
pelo m\'etodo {\it next\/}, da classe {\it Scanner\/}.

A leitura deve terminar quando o texto lido (valor retornado por {\it
next\/}) for nulo.

Uma palavra \'e pal\'{\i}ndrome se ela \'e igual ao seu reverso. Por
exemplo, {\tt \symbol{34}a\symbol{34}}, {\tt
\symbol{34}aba\symbol{34}} e {\tt \symbol{34}abcba\symbol{34}} s\~ao
pal\'{\i}ndromes, enquanto {\tt \symbol{34}ab\symbol{34}} n\~ao \'e.

Voc\^e deve declarar e usar um m\'etodo que testa se um valor de tipo
{\it String\/} \'e pal\'{\i}ndrome.

\end{enumerate}

\end{document}