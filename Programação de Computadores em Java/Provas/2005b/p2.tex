\documentclass{article}

\usepackage[latin1]{inputenc}
\usepackage[portuguese]{babel}

\title{2$^{\underline{\mbox{a}}}$ Provinha de AEDS1}

\author{Professor: Carlos Camar\~ao}

\date{8 de Novembro de 2005}

\addtolength\textwidth{15mm}
\addtolength\textheight{2cm}
\addtolength\topmargin{-2cm}
\setlength{\parindent}{0pt}

\pagestyle{empty}

\begin{document}

\maketitle

Um numeral � um {\em pal�ndromo\/} se, lido da esquerda para a
direita, � o mesmo que quando lido da direita para a esquerda (em
outras palavras, se ele \'e igual ao seu reverso). Por exemplo, 74547
\' um pal\'{\i}ndromo.

\vspace*{\baselineskip}
Um n�mero pode ter uma representa��o que � um pal�ndromo em uma base,
e n�o � em outra. Por exemplo, 17 n�o � um pal�ndromo, mas 10001 �, e
ambas representam o mesmo n�mero se as bases s�o 10 e 2,
respectivamente.

\vspace*{\baselineskip}
Fa�a um programa que leia dois n�meros inteiros positivos $n$ e $m$,
uma lista de $n$ numerais na base 10 e imprima, para cada numeral $i$
dessa lista, uma das seguintes mensagens, conforme apropriado:

  \begin{itemize}
 
     \item ``A representa\c{c}\~ao de $i$ \'e um pal\'{\i}ndromo nas
     bases: $b_1, \ldots, b_k$'' (onde $b_1, \ldots, b_k$ s\~ao as
     bases de 2 a $m$ nas quais a representa\c{c}\~ao de $i$ \'e um
     pal\'{\i}ndromo);

     \item ``A representa\c{c}\~ao de $i$ n\~ao \'e um pal\'{\i}ndromo
     em nenhuma das bases de 2 a $m$''.

  \end{itemize}

Para leitura de $n$ e $m$ e para leitura de cada um dos numerais
inteiros na base 10 voc� pode usar (supor que existe definido) um
m�todo {\tt l�Inteiro} (m�todo est�tico, sem par�metros, que l� um
valor inteiro e retorna esse valor).

\end{document}