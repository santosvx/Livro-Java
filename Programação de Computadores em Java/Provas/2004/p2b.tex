\documentstyle{article}

\title{2$^{\underline{\mbox{a}}}$ Prova de Algoritmos e Estruturas de Dados I}

\pagestyle{empty}	

\author{Professor: Carlos Camar\~ao}
\date{27 de Janeiro de 2004}

\begin{document}

\maketitle
\thispagestyle{empty}

\begin{enumerate}

\item (3 pontos) Escreva um programa que leia um valor inteiro
positivo $n$, calcule o valor de $\sum_{i=1}^n (-i+1)/(i*i)$, e imprima
esse valor. 

O programa (isto \'e, o m\'etodo de leitura) pode supor que a cadeia
de caracteres lida representa um n\'umero inteiro positivo (em outras
palavras, ele n\~ao precisa testar se a cadeia de caracteres lida
representa de fato um n\'umero inteiro positivo).  

\item (2 pontos) Escreva outra vers\~ao para o seu programa acima, que
difere da anterior da seguinte maneira. Se o seu programa usou um
comando de repeti\c{c}\~ao, use agora um m\'etodo recursivo, e
vice-versa (se o seu programa usou um m\'etodo recursivo, use agora um
m\'etodo n\~ao recursivo, contendo um comando de repeti\c{c}\~ao para
implementa\c{c}\~ao do somat\'orio).

\item (5 pontos) Fa\c{c}a um programa para ler um valor inteiro
positivo $n$, e em seguida $n$ seq\"u\^encias de valores inteiros
positivos, cada uma delas terminada por um valor inteiro negativo ou
nulo, e imprimir a soma dos maiores valores de cada uma das
seq\"u\^encias.

Por exemplo, se os valores digitados pelo usu\'ario forem:
  
  \[ \begin{array}{l}
        3    \\
        1    \\
        10   \\
        0    \\
        2    \\
        20   \\
        10   \\
        -1   \\
        3    \\ 
        30   \\
        5    \\
        1    \\
        0    
      \end{array} \]
o programa deve imprimir 50 como resultado (50 = 10+20+30).

O programa pode supor, em cada situa\c{c}\~ao de entrada de dados, que
o valor lido \'o esperado (por exemplo n\~ao \'e fornecida como
entrada nenhuma cadeia de caracteres que n\~ao representa um n\'umero
inteiro). Por exemplo, n\~ao \'e necess\'ario testar se o primeiro
valor lido (de $n$) representa de fato um valor positivo.

\item (5 pontos) Escreva outra vers\~ao para o seu programa acima, que
difere da anterior da seguinte maneira. Se o seu programa declarou e
usou um arranjo, fa\c{c}a uma vers\~ao que n\~ao declara e nem usa um
arranjo. E vice-versa (se o seu programa n\~ao declarou e usou um
arranjo, fa\c{c}a uma vers\~ao que declara e usa um arranjo).

\item (5 pontos) Fa\c{c}a um programa que leia dois n\'umeros inteiros
positivos, $n$ e $m$, e $n$ seq\"u\^encias de $m$ valores inteiros
positivos, cada uma dessas seq\"u\^encias terminada por um valor
inteiro negativo ou zero, e imprima, para $i$ variando de 1 a $m$, a
soma dos $i$-\'esimos inteiros de cada uma das $n$ sequ\^encias.

Por exemplo, se os valores fornecidos como entrada forem:
  
  \[ \begin{array}{l}
        3    \\
        2    \\
        10   \\
        5    \\
        0    \\
        2    \\
        20   \\
        -1   \\
        3    \\
        30   \\
        0
      \end{array} \]
o programa deve imprimir 70 como resultado: 
  70 = (10+5) + (2+20) + (3+30) 

O programa pode super que cada um dos valores fornecidos como entrada
representa um valor esperado (por exemplo, n\~ao \'e preciso testar se
\'e fornecida como entrada uma cadeia de caracteres que n\~ao
representa um n\'umero inteiro, ou se o n\'umero de inteiros positivos
de cada seq\"u\^encia n\~ao \'e o mesmo).

\end{enumerate}

\end{document}

