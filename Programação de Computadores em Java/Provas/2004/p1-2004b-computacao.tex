\documentclass{article}

\usepackage[latin1]{inputenc}

\title{1$^{\underline{\mbox{a}}}$ Prova de Programa\c{c}\~ao de Computadores}

\author{Professor: Carlos Camar\~ao}

\date{19 de Outubro de 2004}

\addtolength\textwidth{15mm}
\setlength{\parindent}{0pt}

\pagestyle{empty}

\begin{document}

\maketitle

\begin{enumerate}
 
\item (5 pontos) Foi feita uma pesquisa entre os habitantes de uma
regi�o, na qual foram coletados dados de idade, sexo (M/F) e
sal�rio. Fa�a um programa para ler os dados e imprimir:
  
\begin{enumerate}

\item a m�dia de sal�rio do grupo, e dos homens e das mulheres do grupo;

\item a maior e menor idade do grupo;

\item a quantidade de homens com 60 anos ou mais com sal�rio maior que
R\$1000,00.

\end{enumerate}

Suponha que os dados s�o lidos do dispositivo de entrada padr�o, e que
existem m�todos {\tt EntradaPadr�o.l�Inteiro()} e {\tt
EntradaPadr�o.l�PontoFlutuante()} que l�em e retornam um valor inteiro
e um valor de ponto flutuante, respectivamtente, desse
dispositivo. Mais precisamente, voc� pode supor que existe uma classe
{\tt EntradaPadr�o} na qual s�o definidos m�todos com ``assinatura''
{\tt public static int l�Inteiro()} e {\tt public static float
l�PontoFlutuante}. Esses m�todos podem ser chamados usando,
respectivamente, {\tt EntradaPadr�o.l�Inteiro()} e \\ {\tt
EntradaPadr�o.l�PontoFlutuante()}.

Voc� pode supor a exist�ncia dessas fun��es tamb�m nas quest�es a
seguir.

Dica: Use Integer.MAX\_VALUE como valor inicial da menor altura.

\item (5 pontos) Escreva em Java uma fun\c{c}\~ao (m\'etodo
est\'atico) que, dado um valor $n$ n\~ao-negativo, calcule o
somat\'orio dos $n$ primeiros termos da seq\"u\^encia $(1-2.0/3),\:
(4+5.0/6,)\: (7-8.0/9),\: (10+11.0/12),\: (13-14.0/15), \ldots$ Se $n$
for igual a 0 (zero), o resultado do somat\'orio \'e igual a 0 (zero).

Fa\c{c}a um programa que leia repetidamente um valor inteiro, at\'e
que o valor lido seja menor que zero, e chame a fun\c{c}\~ao acima
para cada um dos valores n�o-negativos lidos, imprimindo o resultado
calculado pela fun\c{c}\~ao.

\item (5 pontos) Escreva um programa para imprimir tri\^angulos como o
mostrado abai\-xo.

\begin{center}
  \begin{tabular}{c}
      * \\
     ***\\
    *****\\
   *******\\
  *********\\
 ***********\\
*************
  \end{tabular}
\end{center}

O programa deve ler valores inteiros positivos (um a um) e imprimir,
para cada valor lido $n$, um tri�ngulo com n\'umero de linhas igual a
$n$.  Se o valor for menor ou igual a zero, a execu��o do programa
deve terminar.

\item (5 pontos) Escreva um programa que para imprimir o valor de:

  \[ S_n = \sum_{i=0}^n \frac{1}{i!} \]

O seu programa deve ler valores inteiros n�o-negativos e imprimir,
para cada valor lido $n$, o valor de $S_n$. Se $n$ for menor que zero,
a execu��o do programa deve terminar.

N�o use uma fun��o para calcular o fatorial de $i$, para cada termo
($\frac{1}{i!}$). Em vez disso, calcule o fatorial com base no
fatorial do termo anterior, ou seja, use o fato de que $i! = (i-1)!
\times i$ (e $0! = 1$).

\end{enumerate} 

\end{document}