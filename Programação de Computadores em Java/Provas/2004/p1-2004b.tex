\documentclass{article}

\usepackage[latin1]{inputenc}

\title{1$^{\underline{\mbox{a}}}$ Prova de Programa\c{c}\~ao de Computadores}

\author{Professor: Carlos Camar\~ao}

\date{14 de Outubro de 2004}

\addtolength\textwidth{15mm}
\setlength{\parindent}{0pt}

\pagestyle{empty}

\begin{document}

\maketitle

\begin{enumerate}
 
\item (7 pontos) Escreva um programa que leia um valor inteiro
positivo $n$ e imprima em uma linha os valores de cada um dos $n$
primeiros n�meros perfeitos, separados um do outro por um ou mais
espa�os. Um n�mero perfeito � um n�mero inteiro positivo que � igual �
soma dos seus divisores. Exemplo: {\tt 6=1+2+3} e {\tt 28=1+2+4+7+14}
s�o n�meros perfeitos.

Dica: voc� pode ou imprimir diretamente os n�meros perfeitos
encontrados ou concaten�-los em uma cadeia de caracteres (i.e.~em um
``string'') e depois imprimir a cadeia de caracteres (n�o esquecendo
de separar um n�mero do seguinte por um ou mais espa�os).

\item (6 pontos) Escreva um programa para contagem de votos em uma
elei��o presidencial, da qual participam dois candidatos.  Os dados de
entrada do programa representam os votos dos eleitores, codificados
como a seguir. Um valor 1 representa um voto para o primeiro
candidato, e um valor 2 para o segundo candidato; 3 representa um voto
nulo e 4 representa um voto em branco.

O seu programa deve ler os dados de entrada, at� que um valor 0 (zero)
seja lido, e calcular e imprimir o total de votos para cada candidato,
a porcentagem de votos nulos e a porcentagem de votos em branco.

Suponha que os dados s�o lidos do dispositivo de entrada padr�o. Voc�
pode supor que existe um m�todo {\tt EntradaPadr�o.l�Inteiro()} que l�
e retorna um valor inteiro lido desse dispositivo. Mais precisamente,
voc� pode supor que existe uma classe {\tt EntradaPadr�o} na qual �
definido um m�todo com ``assinatura'' {\tt public static int
l�Inteiro()}. Esse m�todo pode ser chamado simplesmente usando {\tt
EntradaPadr�o.l�Inteiro()}.

\item (7 pontos) Escreva um programa que leia repetidamente valores inteiros 
positivos $n$ e $h$, um ap�s o outro, o primeiro representando o
n�mero de um aluno, e o segundo representando a sua altura em
cent�metros, at� que um valor inteiro n�o positivo (para o n�mero ou
para a altura) seja lido. O programa deve determinar qual � o aluno
mais alto e o mais baixo, imprimindo o n�mero do aluno mais alto e sua
altura, e o n�mero do aluno mais baixo e sua altura.

Dica: Use Integer.MAX\_VALUE como valor inicial da menor altura e
n�mero do aluno com menor altura.

%\item (5 pontos) Escreva em Java uma fun\c{c}\~ao (m\'etodo
%est\'atico) que, dado um valor $n$ n\~ao-negativo, calcule o
%somat\'orio dos $n$ primeiros termos da seq\"u\^encia $1, -2*3, 4,
%5*6, 7, -8*9, 10, 11*12, 13, -14*15, \ldots$ Se $n$ for igual a 0
%(zero), o resultado do somat\'orio \'e igual a 0 (zero).
%
%Fa\c{c}a um programa que leia repetidamente um valor inteiro, at\'e
%que o valor lido seja menor que zero, e chame a fun\c{c}\~ao acima
%para cada um dos valores n�o-negativos lidos, imprimindo o resultado
%calculado pela fun\c{c}\~ao.

\end{enumerate}

\end{document}