\documentstyle{article}

\title{Exampe Especial\\ Algoritmos e Estruturas de Dados I}

\pagestyle{empty}	

\addtolength\textwidth{15mm}
\addtolength\textheight{2cm}
\addtolength\topmargin{-2cm}
\setlength{\parindent}{0pt}

\author{Prof.: Carlos Camar\~ao}
\date{5 de Janeiro de 2001}

\begin{document}

\maketitle
\thispagestyle{empty}

Em todas as quest\~oes, voc\^e pode supor a exist\^encia de m\'etodos
est\'aticos de entrada e sa\'{\i}da de valores de tipos b�sicos,
definidos em uma classe {\tt EntradaPadr\~ao}. Por exemplo, para ler
inteiros, voc\^e pode usar {\tt EntradaPadr\~ao.l\^eInteiro()}.

\begin{enumerate}

\item Escreva um programa que leia um valor inteiro positivo $n$, 
calcule o valor dado por $\sum_{i=1}^n (-(i+1)/(i*i))$, e imprima esse
valor.  Decomponha o seu programa em tr\^es partes, cada uma delas
implementada por um m\'etodo: leitura, c\'alculo do valor desejado a
partir do valor lido, e impress\~ao do valor calculado.

O m\'etodo de leitura deve ler uma cadeia de caracteres e causar uma
exce\c{c}\~ao se essa cadeia n\~ao representar um n\'umero inteiro
positivo. Se a exce\c{c}\~ao ocorrer, ela deve ser tratada no programa
principal. Esse tratamento deve apenas emitir uma mensagem de erro
apropriada, e em seguida a execu\c{c}\~ao do programa deve ser
interrompida.

\item Fa\c{c}a um programa para ler um valor inteiro positivo $n$, e
em seguida $n$ seq\"u\^encias de valores inteiros positivos, cada uma
delas terminada por um valor inteiro negativo ou nulo, e imprimir a
soma dos maiores valores de cada uma das seq\"u\^encias.

Por exemplo, se os valores digitados pelo usu\'ario forem:
  
  \[ \begin{array}{l}
        5    \\
        2    \\
        10   \\
        0    \\
        2    \\
        18   \\
        -1   \\
        3    \\ 
        30   \\
        5    \\
        1    \\
        0    
      \end{array} \]

o programa deve imprimir como resultado o valor 76 (pois 76 = (5+2+10) +
(2+18) + (3+30+5+1)).

Em cada situa\c{c}\~ao de entrada de dados, se um valor n\~ao esperado
for fornecido com entrada (por exemplo uma cadeia de caracteres que
n\~ao representa um n\'umero inteiro), uma exce\c{c}\~ao deve ser
causada, e o processo de entrada de dados pelo usu\'ario do programa
deve recome\c{c}ar, a partir do in\'{\i}cio (isto \'e, a partir da
leitura de $n$).

Dica 1: o m\'etodo {\it parseInt\/} causa uma exce\c{c}\~ao ({\it
NumberFormatException\/}) se a cadeia de caracteres fornecida como
argumento n\~ao representar um n\'umero inteiro.

Dica 2: N\~ao esque\c{c}a de testar se o primeiro valor lido
\'e um valor positivo, e causar uma exce\c{c}\~ao em caso contr\'ario. 
Voc\^e pode causar e tratar apenas a exce\c{c}\~ao {\it Exception\/},
pois a exce\c{c}\~ao {\it NumberFormatException\/} \'e um objeto da
classe {\it Exception\/}, e portanto ser\'a tamb\'em tratada por um
tratador que trata (recebe como argumento) uma exce\c{c}\~ao {\it
NumberFormatException\/}.

\item Escreva outra vers\~ao para o seu programa acima, que difere da
anterior da seguinte maneira. Se o seu programa declarou e usou um
arranjo, fa\c{c}a uma vers\~ao que n\~ao declara e nem usa um
arranjo. E vice-versa (se o seu programa n\~ao declarou e usou um
arranjo, fa\c{c}a uma vers\~ao que declara e usa um arranjo).

\item (7 pontos) Uma opera\c{c}\~ao de {\em contra\c{c}\~ao de uma
lista\/} \'e definida como sendo tal que, dados um inteiro positivo
$i$, chamado de {\it \'{\i}ndice\/} da contra\c{c}\~ao, e uma lista de
$n$ n\'umeros inteiros $x = [a_1, a_2, \ldots, a_i,
\ldots, a_n]$, onde sup\~oe-se $i < n$, retorna como resultado a
lista $[a_1, \ldots, a_{i-1}, a_i-a_{i+1}, a_{i+2},\ldots, a_n]$. 

Aplicando $n-1$ contra\c{c}\~oes a uma lista de $n$ inteiros quaisquer
obt\'em-se uma lista com um \'unico inteiro, chamado de {\em valor
alvo\/}. Por exemplo, se chamarmos de {\it con\/} a opera\c{c}\~ao de
contra\c{c}\~ao e por $con(x,i)$ a opera\c{c}\~ao de aplicar {\it
con\/} a uma lista {\it x} e a um \'{\i}ndice $i$, temos:

  \[ \begin{array}{rcl}
       con([12,10,4,3,5],2) & = & [12,6,3,5] \\
       con([12,6,3,5],3)    & = & [12,6,-2]\\
       con([12,6,-2],2)     & = & [12,8]\\
       con([12,8],1)        & = & [4]
     \end{array}
  \]

Fa\c{c}a um programa que:

\begin{itemize}

\item leia um valor inteiro positivo $n$ (o n\'umero de inteiros da
seq\"u\^encia original), $n$ valores inteiros de uma seq\"u\^encia
$a_1, \ldots, a_n$ e, em seguida, uma seq\"u\^encia de $n-1$ valores
que indicam os \'{\i}ndices de constra\c{c}\~oes, 

\item realize contra\c{c}\~oes consecutivamente como no exemplo
acima, e

\item imprima o valor alvo $x$. 

\end{itemize}

Se n\~ao existir valor alvo (i.e.~se a entrada estiver de alguma forma
incorreta), o programa deve emitir uma mensagem de erro apropriada.

Dica: Use um contador para indicar o n\'umero de valores v\'alidos
para a seq\"u\^encia, inicialmente igual a $n$ (tamanho da
seq\"u\^encia original) e decrementado de 1 a cada contra\c{c}\~ao.

\end{enumerate}

\end{document}

